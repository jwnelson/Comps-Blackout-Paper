\documentclass[twocolumn]{article}



\begin{document}
%opening
\title{Methods of mitigating the electromagnetic blackout effect of plasma sheaths}
\author{Jack Nelson,\\
	Physics Department,\\
	Occidental College, Los Angeles, CA}
\date{October 2016}
\twocolumn[ % hacky trick to get a one-column abstract in a two-column article
	\begin{@twocolumnfalse}
		\maketitle
			\begin{abstract}
			Airborne vehicles traveling through an atmosphere at hypersonic velocities, especially in excess of Mach 15, face significant atmospheric heating that produces plasma sheaths which surround the vehicle.
			 Depending on the density of the atmosphere, the shape of the vehicle, and its speed, free electrons in the plasma attenuate and reflect incident electromagnetic waves, making communication to and from the hypersonic vehicle difficult or impossible.
			 Various methods of communicating through the plasma sheath have been proposed and studied since the early 1960's, yet a single elegant solution has not been found in unclassified research to date.
			 
			 In this paper, we will establish a theoretical overview of the problem and look at the associated plasma physics in order to better frame the potential solutions.
			 These solutions can be divided into two types: passive methods, which seek to affect the properties of the plasma sheath using passive features such the shaping of the vehicle, and active methods, which use active systems to communicate through the plasma sheath barrier.
			 We will examine some of the passive and active solutions proposed since research into the subject began in the early 60's.
			 Recent studies which propose solving the communications blackout problem using active electromagnetic fields and $\vec{E}\times\vec{B}$ techniques are also evaluated in the context of the earlier methods.
			 A method which uses electromagnetic scattering off of resonances induced by radio signals from outside the plasma sheath is presented as a solution that has the potential to minimize weight and equipment volume in the vehicle, yet remain effective.
	 
	
			\end{abstract}
		\end{@twocolumnfalse}
	]
\section{Introduction}
	Spacecraft reentering an atmosphere from orbital velocities and aircraft traveling at hypersonic speeds experience the problem of aerodynamic heating, in which their large kinetic energies are transferred into heat by hypersonic interactions with the atmosphere.
	The problem of convective aerodynamic heating of a reentry vehicle was adequately solved by the first human space flights in the early 1960's.
	For reentry speeds, was found that blunt bodies as opposed to slender, streamlined ones minimize the total heat which is transferred to the vehicle from the air. \cite{allen_study_1958}
	Blunt body capsule designs coupled with ablative heat shield technology created a simple yet effective solution to the reentry heating problem which is still widely used today.
	For hypersonic aircraft which operate at less than reentry speeds, such as NASA's X-15 piloted research vehicle, the heating problem is solved using thick skin heat sinks made of high heat capacity superalloys such as Inconel-X. \cite{stillwell_x-15_1965}
	
	A side-effect of the hypersonic heating problem is the generation of large volumes of plasmisized atmospheric gases and the formation of a plasma sheath which envelopes the vehicle at high speeds.
	The basic properties of plasma sheaths, such as thickness, temperature, and density, are not uniform over the entire sheath, are highly dependent on the vehicle configuration and atmosphere, and are extremely difficult to accurately predict.
	
\section{Hypersonic Plasma Sheaths}

\subsection*{Plasma Waves}

\section{Overview of Solutions}
\subsection*{Passive Techniques}
\subsection*{Active Techniques}

\section{Prior Research}
	\subsection*{NASA RAM Studies}
	Radio wave attenuation in plasma sheaths and various methods of preventing it have been studied since at least the early 1960's.
	
	NASA's Radio Attenuation Measurement (RAM) project took place over nearly a decade from 1961 until 1970.
	Project RAM was one of the earliest sustained research efforts specifically targeted at the radio blackout problem for reentering space vehicles, and also the only known flight tests to date specifically to study the phenomenon and potential solutions to it.
	Since the scope of study was specifically targeted towards the radio blackout problem, and no other comparable flight testing has followed, data and findings from the RAM program are still cited in literature on the subject today.
	
	The RAM program consisted of a series of three flight test campaigns and eight flights total.
	The tests consisted of launching a reentry vehicle packed with plasma and radio diagnostic experiments on three and four-staged sounding rockets on a sub-orbital trajectory.
	The RAM A and B campaigns were comprised of four test flights which studied the reentry plasma sheath at a reentry velocity of 5.5 km/s (18,000 ft/s). \cite{huber_entry_1971}
	The two RAM A flights primarily studied aerodynamic shaping and magnetic window techniques, while the three RAM B flights studied water quenching and advanced study of the plasma sheath.
	The four RAM C flights reentered at 7.6 km/s (25,000 ft/s), which is comparable to low Earth orbit reentry velocities of around 8 km/s. 
	The RAM C flights studied advanced plasma diagnostics as well as liquid quenching. \cite{huber_entry_1971}

\section{Active Solutions}
\subsection*{Lorentz Windows}
\subsection*{Stokes Waves}

\bibliography{plasma_blackout}
\bibliographystyle{ieee}

\end{document}
