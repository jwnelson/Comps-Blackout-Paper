\documentclass[twocolumn]{article}



\begin{document}
%opening
\title{Methods of mitigating the electromagnetic blackout effect of plasma sheaths}
\author{Jack Nelson,\\
	Occidental College Physics Department,\\
	Occidental College, Los Angeles, CA}
\date{October 2016}
\twocolumn[ % hacky trick to get a one-column abstract in a two-column article
	\begin{@twocolumnfalse}
		\maketitle
			\begin{abstract}
	Airborne vehicles traveling through an atmosphere at hypersonic velocities, especially in excess of Mach 15, face significant heating that produces plasma sheaths which surround the vehicle.
	 Depending on the density of the atmosphere, the shape of the vehicle, and its speed, free electrons in the plasma attenuate and even reflect incident electromagnetic waves, making communication to and from the hypersonic vehicle difficult or impossible. 
	 Various different methods of communicating through the plasma sheath have been investigated since the early 1960's, yet a single elegant solution has not been found. 
	 In this paper, some of the most promising solutions are evaluated and their respective research reviewed.
	 An analytical overview of the problem is presented in order to better frame the potential solutions.
	 Recent studies which propose solving the communications blackout problem using active electromagnetic fields and $\vec{E}\times\vec{B}$ techniques are also evaluated in the context of the earlier methods.
	
			\end{abstract}
		\end{@twocolumnfalse}
	]
\section{Introduction}

\section{Plasma Sheath Physics}
\subsection{Hypersonic Shockwaves}
\subsection{Plasma Generation}


\section{Effects of Vehicle Shape on the Plasma Sheath}
\subsection{Narrow Bodies}
\subsection{Blunt Bodies}


\section{Overview of Solutions}
\subsection{Passive Techniques}
\subsection{Active Techniques}

	
\section{Early Research}
\subsection{NASA RAM Studies}
\subsection{Gemini and Apollo Missions}
\subsection{Space Shuttle}

\section{Active Solutions}
\subsection{Lorentz Windows}



\end{document}
