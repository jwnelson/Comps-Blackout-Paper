\documentclass[twocolumn]{article}



\begin{document}
%opening
\title{Active methods of mitigating the electromagnetic blackout effect of plasma sheaths}
\author{Jack Nelson,\\
	Physics Department,\\
	Occidental College, Los Angeles, CA}
\date{October 2016}
\twocolumn[ % hacky trick to get a one-column abstract in a two-column article
	\begin{@twocolumnfalse}
		\maketitle
			\begin{abstract}
			Airborne vehicles traveling through an atmosphere at hypersonic velocities, especially in excess of Mach 15, face significant atmospheric heating that produces plasma sheaths which surround the vehicle.
			 Depending on the density of the atmosphere, the shape of the vehicle, and its speed, free electrons in the plasma attenuate and reflect incident electromagnetic waves, making communication to and from the hypersonic vehicle difficult or impossible.
			 Various methods of communicating through the plasma sheath have been proposed and studied since the early 1960's, yet a single elegant solution has not been found in unclassified research to date.
			 
			 In this paper, we will establish a theoretical overview of the problem and look at the associated plasma physics in order to better frame the potential solutions.
			 These solutions can be divided into two types: passive methods, which seek to affect the properties of the plasma sheath using passive features such the shaping of the vehicle, and active methods, which use active systems to communicate through the plasma sheath barrier.
			 We will examine some of the passive and active solutions proposed since research into the subject began in the early 60's.
			 Recent studies which propose solving the communications blackout problem using active electromagnetic fields and $\vec{E}\times\vec{B}$ techniques are also evaluated in the context of the earlier methods.
			 A method which uses electromagnetic scattering off of resonances induced by radio signals from outside the plasma sheath is presented as a solution that has the potential to minimize weight and equipment volume in the vehicle, yet remain effective.
	 
	
			\end{abstract}
		\end{@twocolumnfalse}
	]
\section{Introduction}
	The radio communications blackout problem caused by atmospheric reentry was first realized during early Intercontinental Ballistic Missile (ICBM) tests in the 1950's.
	
\section{Plasma Sheath Physics}
\subsection{Hypersonic Shockwaves}
\subsection{Plasma Generation}


\section{Effects of Vehicle Shape on the Plasma Sheath}
\subsection{Narrow Bodies}
\subsection{Blunt Bodies}


\section{Overview of Solutions}
\subsection{Passive Techniques}
\subsection{Active Techniques}

	
\section{Early Research}
\subsection{NASA RAM Studies}
\subsection{Gemini and Apollo Missions}
\subsection{Space Shuttle}

\section{Active Solutions}
\subsection{Lorentz Windows}



\end{document}
