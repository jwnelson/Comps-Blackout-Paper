\documentclass[]{tufte-handout}

%opening
\title{The Plasma Blackout Problem Equation Cheat Sheet}
\author{Jack Nelson}
\date{October 2016}


\usepackage{graphicx} % allow embedded images
\setkeys{Gin}{width=\linewidth,totalheight=\textheight,keepaspectratio}
\graphicspath{{graphics/}} % set of paths to search for images
\usepackage{amsmath}  % extended mathematics
\usepackage{booktabs} % book-quality tables
\usepackage{units}    % non-stacked fractions and better unit spacing
\usepackage{multicol} % multiple column layout facilities
\usepackage{lipsum}   % filler text
\usepackage{fancyvrb} % extended verbatim environments
\fvset{fontsize=\normalsize}% default font size for fancy-verbatim environments

\begin{document}

\maketitle

%\begin{abstract}
%\end{abstract}

\section{Intro to the Blackout Problem}

\section{Lorentz Windows}
These are the equations used by Kim et. al. in the computer modeling of a plasma sheath in the vicinity of an $\mathbf{E}\times\mathbf{B}$ field~\cite{kim_analysis_2008}.

\textbf{Ion Transport Equations}

\begin{equation}
	\nabla \cdot (\mathbf{V_i} n) = 0
\end{equation}
\begin{equation}
	m_{ i }n(\mathbf { V_{ i } } \cdot \nabla \mathbf { V_{ i } } )=en(\mathbf { E } +\mathbf { V_{ i } } \times \mathbf { B } )-m_{ i }n\nu _{ c }\mathbf { V_{ i } }
\end{equation}

\textbf{Current Density of Plasma Sheath}
\begin{equation}
	\mathbf{j} = \sigma \left(\mathbf{E} + \frac{kT_e}{e}\nabla \ln n - \frac{\mathbf{j} \times \mathbf{B}}{en} + \left(\mathbf{V_i} \times \mathbf{B}\right)\right)
\end{equation}

\textbf{Current Density Conservation}
\begin{equation}
\nabla \cdot \mathbf{j} = 0
\end{equation}

\bibliography{plasma_blackout}
\bibliographystyle{ieeetr}

\end{document}
